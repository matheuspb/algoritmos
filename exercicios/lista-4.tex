\documentclass{article}
\usepackage[a4paper, margin=3cm]{geometry}
\usepackage[brazilian]{babel}
\usepackage[utf8]{inputenc}
\usepackage{amsmath, amssymb, amsthm, listings}

\title{Projeto e Análise de Algoritmos -- Lista de Exercícios 4}
\author{Lucas Bordignon \and Matheus Bittencourt \and Vinicius Macelai}
\date{}

\begin{document}

\maketitle

\begin{enumerate}
	%%%% exercício 1 %%%%
	\item \lstinputlisting[language=Python, frame=single, tabsize=4]
		{../algoritmos/lista-4/remove_duplicates.py}

	%%%% exercício 2 %%%%
	\item \lstinputlisting[language=Python, frame=single, tabsize=4]
		{../algoritmos/lista-4/check_ii.py}

	%%%% exercício 3 %%%%
	\item \begin{enumerate}

		%%% item a) %%%
		\item A primeira operação vai tomar $2n$ comparações, a segunda $3n$ e
		assim suscetivamente, logo:

		\begin{align*}
		T(n) &= \sum_{i=2}^{k}i\cdot n \\
		&= n\left(\frac{k^2 + k - 2}{2}\right) \\
		&= \Theta(nk^2)
		\end{align*}

		%%% item b) %%%
		\item Assumindo que $k$ seja uma potência de 2, podemos fazer o
		\textit{merge} dos pares de vetores obtendo agora $k/2$ vetores de
		tamanho $2n$ e repetir isso até que tenhamos apenas 1 vetor de tamanho
		$kn$. Em cada passo desse algoritmo iremos executar $kn$ comparações e
		iremos executar esse passo $\log k$ vezes, logo:

		$$ T(n) = \Theta(kn \log k) $$

	\end{enumerate}

	%%%% exercício 4 %%%%
	\item

\end{enumerate}

\end{document}

