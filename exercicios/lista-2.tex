\documentclass{article}
\usepackage[a4paper, margin=3cm]{geometry}
\usepackage[brazilian]{babel}
\usepackage[utf8]{inputenc}
\usepackage{amsmath, amssymb, amsthm, listings}

\title{Projeto e Análise de Algoritmos -- Lista de Exercícios 2}
\author{Lucas Bordignon \and Matheus Bittencourt \and Vinicius Macelai}
\date{}

\begin{document}

\maketitle

\begin{enumerate}
	%%%% exercicio 1 %%%%
	\item

	%%%% exercicio 2 %%%%
	\item \begin{enumerate}
		%%%% a) %%%%
		\item $f(n) \in O(g(n))$:
		$$ 0 \le n-100 \le c(n-200) $$
		A inequação se verifica para $n\ge300$ e $c=2$.

		$f(n) \in \Omega(g(n))$:
		\begin{align*}
			0 &\le n-200 \le n-100 \\
			0 &\le n \le n+100 \\
		\end{align*}

		Logo, $f(n) \in \Theta(g(n))$.

		%%%% b) %%%%
		\item $f(n) \in O(g(n))$:
		\begin{align*}
			0 &\le n^{1/2} \le cn^{2/3} \\
			0 &\le n^{1/2-2/3} \le c \\
			0 &\le n^{3/6-4/6} \le c \\
			0 &\le n^{-1/6} \le c
		\end{align*}
		Basta escolher um $c$ que satisfaça a inequação acima com $n=n_0$

		$f(n) \notin \Omega(g(n))$:
		\begin{align*}
			0 &\le cn^{2/3} \le n^{1/2} \\
			0 &\le c \le n^{-1/6}
		\end{align*}
		A inequação não se verifica quando $n > c$. Logo, $f(n) \notin
		\Theta(g(n))$.

		%%%% c) %%%%
		\item $f(n) \in O(g(n))$:
		\begin{align*}
			0 &\le 100n + \log n \le c (n + \log^2 n) \\
			0 &\le \frac{100n + \log n}{n + \log^2 n} \le c
		\end{align*}
		Visto que $\lim_{x\to\infty}\frac{100n + \log n}{n + \log^2 n}=100$,
		então basta que $c>100$ que a inequação será satisfeita para todo
		$n>n_0$.

		$f(n) \in \Omega(g(n))$:
		\begin{align*}
			0 &\le c (n + \log^2 n) \le 100n + \log n \\
			0 &\le c \le \frac{100n + \log n}{n + \log^2 n}
		\end{align*}
		Agora basta escolher $c<100$. Logo, $f(n) \in \Theta(g(n))$.

		%%%% d) %%%%
		\item $f(n) \in O(g(n))$:
		\begin{align*}
			0 &\le n \log n \le 10n \log 10n \\
			0 &\le \log n \le 10 (\log 10 + \log n)
		\end{align*}

		$f(n) \in \Omega(g(n))$:
		\begin{align*}
			0 &\le c10n \log 10n \le n \log n \\
			0 &\le c10 \log 10n \le \log n \\
			0 &\le c \le \frac{\log n}{10 \log 10n} \\
			0 &\le c \le \frac{\log n}{10 (\log 10 + \log n)}
		\end{align*}
		A parte da direita sempre está crescendo (semelhante com o que acontece
		na função $f(x)=\frac{x}{1+x}$), então basta escolher um $c$ pequeno o
		suficiente que a inequação é satisfeita para $n>n_0$. Logo, $f(n) \in
		\Theta(g(n))$.

		%%%% e) %%%%
		\item $f(n) \in O(g(n))$:
		\begin{align*}
			0 &\le \log 2n \le \log 3n \\
			0 &\le \log 2 + \log n \le \log 3 + \log n \\
			0 &\le \log 2 \le \log 3
		\end{align*}

		$f(n) \in \Omega(g(n))$:
		\begin{align*}
			0 &\le c \log 3n \le \log 2n \\
			0 &\le c \le \frac{\log 2n}{\log 3n}
		\end{align*}
		Visto que, $\lim_{n\to\infty}\frac{\log 2n}{\log 3n}=1$, basta que
		$c<1$ e haverá algum $n_0$ que satisfaça a equação. Logo $f(n) \in
		\Theta(g(n))$.

		%%%% f) %%%%
		\item $f(n) \in O(g(n))$:
		\begin{align*}
			0 &\le 10 \log n \le c \log n^2 \\
			0 &\le 10 \log n \le 2c \log n
		\end{align*}
		Basta que $c>5$.

		$f(n) \in \Omega(g(n))$:
		\begin{align*}
			0 &\le c \log n^2 \le 10 \log n \\
			0 &\le 2c \log n \le 10 \log n
		\end{align*}
		Basta que $c<5$. Logo $f(n) \in \Theta(g(n))$.

		%%%% g) %%%%
		\item $f(n) \notin O(g(n))$:
		\begin{align*}
			0 &\le \frac{n^2}{\log n} \le c n \log^2 n \\
			0 &\le \frac{n}{\log^3 n} \le c
		\end{align*}
		A função  da esquerda é crescente para valores positivos, logo não é
		possível encontrar uma constante $c$ que satisfaça a inequação.

		$f(n) \in \Omega(g(n))$:
		\begin{align*}
			0 &\le c n \log^2 n \le \frac{n^2}{\log n} \\
			0 &\le c \le \frac{n}{\log^3 n} \\
		\end{align*}
		Como a função da direita é crescente, podemos encontrar um $c$ que
		satisfaça a inequação para todo $n > n_0$. Logo $f(n) \notin
		\Theta(g(n))$.

		%%%% h) %%%%
		\item $f(n) \in O(g(n))$:
		\begin{align*}
			0 &\le {n^{0.1}} \le c \log^{10} n \\
			0 &\le \frac{\sqrt[10]{n}}{\log^{10} n} \le c
		\end{align*}
		A função da esquerda é decrescente, já que $g(n)$ cresce mais
		rapidamente que a função no numerador.

		$f(n) \notin \Omega(g(n))$:
		\begin{align*}
			0 &\le c \log^{10} n \le {n^{0.1}}\\
			0 &\le c \le \frac{\sqrt[10]{n}}{\log^{10} n}
		\end{align*}
		Como a função da direita é decrescente, não conseguimos encontrar um
		$c$ que satisfaça a inequação para todo $n > n_0$. Logo $f(n) \notin
		\Theta(g(n))$.

		%%%% i) %%%%
		\item $f(n) \notin O(g(n))$:
		\begin{align*}
			0 &\le {\log ^{\log n} n} \le c \frac{n}{\log n} \\
			0 &\le \frac{\log ^{\log n + 1} n}{n} \le c
		\end{align*}
		A função da esquerda é crescente para valores positivos, logo, não é
		possível encontrar uma constante $c$ que satisfaça a inequação.

		$f(n) \in \Omega(g(n))$:
		\begin{align*}
		0 &\le  c \frac{n}{\log n} \le {\log ^{\log n} n} \\
		0 &\le c \le \frac{\log ^{\log n + 1} n}{n}
		\end{align*}
		Como a função da direita é crescente, podemos encontrar um $c$ que
		satisfaça a inequação para todo $n > n_0$. Logo $f(n) \in
		\Theta(g(n))$.

		%%%% j) %%%%
		\item $f(n) \in O(g(n))$:
		\begin{align*}
			0 &\le \sqrt{n} \le c {\log^3 n} \\
			0 &\le \frac{\sqrt{n}}{{\log^3 n}} \le c
		\end{align*}
		A função $f(n)$ cresce mais lentamente do que a função $g(n)$, $\forall
		n \geq 4$

		$f(n) \notin \Omega(g(n))$:
		\begin{align*}
			0 &\le c {\log^3 n} \le \sqrt{n} \\
			0 &\le c \le \frac{\sqrt{n}}{\log^3 n}
		\end{align*}
		Como a função da direita é decrescente, não é possível encontrar um $c$
		que satisfaça a inequação para todo $n > n_0$. Logo $f(n) \notin
		\Theta(g(n))$.

		%%%% k) %%%%
		\item $f(n) \in O(g(n))$:
		\begin{align*}
			0 &\le \sqrt{n} \le c {5^{\log n}} \\
			0 &\le \frac{\sqrt{n}}{5^{\log n}} \le c
		\end{align*}
		A função da esquerda é uma função decrescente, dado que $5^{\log n}$
		cresce mais rapidamente do que $\sqrt{n}$.  $\forall n \geq 4$

		$f(n) \notin \Omega(g(n))$:
		\begin{align*}
			0 &\le c {5^{\log n}} \le \sqrt{n}\\
			0 &\le c \le \frac{\sqrt{n}}{5^{\log n}}
		\end{align*}
		Como o lado direito está sempre diminuindo, devido ao limite
		apresentado acima, não podemos escolher uma constante $c$ que satisfaça
		a equação, já que sempre haverá um $n$ grande o suficiente que faça com
		que o lado direito seja menor que o esquerdo. Logo, $f(n) \notin
		\Theta(g(n))$

		%%%% l) %%%%
		\item $f(n) \in O(g(n))$:
		\begin{align*}
			0 &\le n2^n \le c 3^n \\
			0 &\le n\left(\frac{2}{3}\right)^n \le c
		\end{align*}
		Visto que $\lim_{n\to\infty}n\left(\frac{2}{3}\right)^n = 0$, então
		basta escolher um $c > 0$ que existirá um $n_0$ que satisfaça a
		inequação.

		$f(n) \notin \Omega(g(n))$:
		\begin{align*}
			0 &\le c 3^n \le n 2^n \\
			0 &\le c \le n \left(\frac{2}{3}\right)^n
		\end{align*}
		Como o lado direito está sempre diminuindo, devido ao limite
		apresentado acima, não podemos escolher uma constante $c$ que satisfaça
		a equação, já que sempre haverá um $n$ grande o suficiente que faça com
		que o lado direito seja menor que o esquerdo. Logo, $f(n) \notin
		\Theta(g(n))$

		%%%% m) %%%%
		\item $f(n) \in O(g(n))$:
		\begin{align*}
			0 &\le 2^n \le c 2^{n+1} \\
			0 &\le \frac{2^n}{2^{n+1}} \le c \\
			0 &\le \frac{1}{2} \le c
		\end{align*}

		$f(n) \in \Omega(g(n))$:
		\begin{align*}
			0 &\le c 2^{n+1} \le 2^n \\
			0 &\le c \le \frac{1}{2}
		\end{align*}

		Logo, $f(n) \in \Theta(g(n))$.

		%%%% n) %%%%
		\item $f(n) \notin O(g(n))$:
		\begin{align*}
			0 \le n! &\le c \cdot 2^n \\
			0 \le n \cdot (n-1) \cdot (n-2) \ldots 2 \cdot 1 &\le c \cdot 2
			\cdot 2 \ldots 2 \cdot 2
		\end{align*}
		A inequação claramente não é satisfeita quando $n \ge c$.

		$f(n) \in \Omega(g(n))$:
		\begin{align*}
			0 \le c \cdot 2 \cdot 2 \ldots 2 \cdot 2 \le n \cdot (n-1) \cdot
			(n-2) \ldots 2 \cdot 1
		\end{align*}
		Basta que $c=1$ para que a inequação seja satisfeita. Logo, $f(n)
		\notin \Theta(g(n))$.

		%%%% o) %%%%
		\item
	\end{enumerate}

	%%%% exercicio 3 %%%%
	\item No exercício 5, tanto no melhor caso quanto no pior, $T(n)$ se trata
	de um polinômio na forma $an + b$, então $T(n) \in \Theta(n)$. No exercício
	6, $T(n)$ está na forma $an^2 + bn + c$ no melhor e no pior caso, logo,
	$T(n) \in \Theta(n^2)$.

	%%%% exercicio 4 %%%%
	\item \begin{enumerate}
		\item O limite inferior deve ser $n$ comparações, pois não temos como
		determinar o $n$-ésimo termo sem ao menos $n$ comparações.

		\item \begin{lstlisting}[tabsize=4]
K-esimo(A[1..n], B[1..n], k):
	for i <- 1 to k - 1 do
		if A[1] > B[1] then
			remove A[1]
		else
			remove B[1]
	return max(A[1], B[1])
		\end{lstlisting}
	\end{enumerate}

\end{enumerate}

\end{document}
