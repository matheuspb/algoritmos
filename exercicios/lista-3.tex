\documentclass{article}
\usepackage[a4paper, margin=3cm]{geometry}
\usepackage[brazilian]{babel}
\usepackage[utf8]{inputenc}
\usepackage{qtree}
\usepackage{amsmath, amssymb, amsthm, listings}

\title{Projeto e Análise de Algoritmos -- Lista de Exercícios 3}
\author{Lucas Bordignon \and Matheus Bittencourt \and Vinicius Macelai}
\date{}

\begin{document}

\maketitle

\begin{enumerate}
	%%%% exercício 1 %%%%
	\item \begin{enumerate}

		%%% item a) %%%
		\item \begin{align*}
			T(n) &= T(n-1) + c \\
			&= T(n-2) + c + c \\
			&\qquad\vdots \\
			&= T(n-k) + \underbrace{\ldots + c + c}_{\text{$k$ vezes}} \\
			&= T(1) + (n-1)\cdot c \\
			&= \Theta(n)
		\end{align*}

		%%% item b) %%%
		\item \begin{align*}
			T(n) &= T(n-1) + 2^n \\
			&= T(n-2) + 2^n + 2^n \\
			&\qquad\vdots \\
			&= T(n-k) + \underbrace{\ldots+2^n+2^n}_{\text{$k$ vezes}} \\
			&= T(0) + n\cdot2^n \\
			&= \Theta(n2^n)
		\end{align*}

		%%% item c) %%%
		\item \begin{align*}
			T(n) &= cT(n-1) \\
			&= c(cT(n-2)) \\
			&\qquad\vdots \\
			&= \underbrace{c(c(c(\cdots c}_{\text{$k$ vezes}}T(n-k)))) \\
			&= c^{n-1}T(1) \\
			&= \Theta(2^n)
		\end{align*}

		%%% item d) %%%
		\item Assumindo que $n=2^k$ e consequentemente $k = \lg n$
		\begin{align*}
			T(n) &= n + 3T(n/2) \\
			&= n + 3(n/2 + 3T(n/4)) \\
			&= n + 3(n/2 + 3(n/4 + 3T(n/8))) \\
			&= n + 3n/2 + 9n/4 + \ldots + 3^{k-1}n/2^{k-1} + 3^k \\
			&= 3^{\lg n}+n\sum_{i=0}^{k-1=\lg n-1}\left(\frac{3}{2}\right)^i \\
			&= 3^{\lg n}+n\left(\frac{(3/2)^{\lg n -1+1}-1}{(3/2)-1}\right) \\
			&= 3^{\lg n} + 2n \left((3/2)^{\lg n}-1\right) \\
			&= n^{\lg 3} + 2nn^{\lg 3/2} - 2n \\
			&= n^{\lg 3} + 2n^{\lg 3/2 + \lg 2} - 2n \\
			&= n^{\lg 3} + 2n^{\lg 3} - 2n \\
			&= 3 n^{\lg 3} - 2n = \Theta(n^{\lg 3})
		\end{align*}

		%%% item e) %%%
		\item Assumindo que $n, k$ são potências de $2$ e $n=2^k$.
		\begin{align*}
			T(n) &= T(n^{1/2}) + \lg n \\
			&= T(n^{1/4}) + \lg n^{1/2} + \lg n \\
			&= T(n^{1/8}) + \lg n^{1/4} + \lg n^{1/2} + \lg n \\
			&= 1 + \lg n^{\frac{1}{k/2}} + \ldots +
				\lg n^{1/4} + \lg n^{1/2} + \lg n
		\end{align*}
		Jogando todos os expoentes para a frente dos $\lg$s:
		\begin{align*}
			T(n) &= 1 + \lg n \sum_{i=1}^{\lg k}\frac{2^i}{k} \\
			&= 1 + \frac{\lg n}{k} \sum_{i=1}^{\lg k}2^i \\
			&= 1 + \frac{\lg n}{k} (2(2^{\lg k} - 1)) \\
			&= 1 + \frac{2 \lg n}{k} (k - 1) \\
			&= 1+2\lg n -\frac{2 \lg n}{\lg n}
				\qquad\text{lembrando que $k=\lg n$} \\
			&= 2\lg n - 1 \\
			&= \Theta(\lg n)
		\end{align*}

	\end{enumerate}

	%%%% exercício 2 %%%%
  \item \begin{enumerate}
      \item
      \begin{LARGE}
         \Tree[.n [.$\frac{n}{2}$ [.$\frac{n}{4}$ ]
                                  [.$\frac{n}{4}$ ]
                                  [.$\frac{n}{4}$ ] ]
                  [.$\frac{n}{2}$ [.$\frac{n}{4}$ ]
                                  [.$\frac{n}{4}$ ]
                                  [.$\frac{n}{4}$ ] ]
                  [.$\frac{n}{2}$ [.$\frac{n}{4}$ ]
                                  [.$\frac{n}{4}$ ]
                                  [.$\frac{n}{4}$ ] ]
                 ]
      \end{LARGE}

      Analisando a árvore, no primeiro nível temos uma complexidade $n$,
      no segundo nível, complexidade $\frac{3n}{2}$, ou seja, $O(n)$.
      No terceiro, $\frac{9n}{4}$. Consequentemente, o problema inicial
      é $O(n)$.

      \item
      \begin{LARGE}
        \Tree[.$n^2$ [.$\frac{n^2}{2}$ [.$\frac{n^2}{4}$ [.$\frac{n^2}{8}$ ] ] ] ]
      \end{LARGE}

      Analisando a árvore, no primeiro nível temos uma complexidade $n^2$,
      no segundo nível, complexidade $\frac{n^2}{2}$, no terceiro,
      $\frac{n^2}{4}$, e assim vai. Consequentemente, a árvore desce em
      proporção logarítmica ($\frac{1}{2}, \frac{1}{4}, \frac{1}{8}, ... $)
      e o problema inicial é $O(n^2)$.

      \item
      \begin{LARGE}
         \Tree[.n [.$n-1$ [.$n-2$ ]
                            [.$\frac{n-1}{2}$ ] ]
                  [.$\frac{n}{2}$ [.$\frac{n-2}{2}$ ]
                                  [.$\frac{n}{4}$ ] ]
                 ]
      \end{LARGE}

      Analisando a árvore, no primeiro nível temos uma complexidade $n$,
      no segundo nível, complexidade $\frac{3n-2}{2}$, ou seja, $O(n)$.
      No terceiro, $\frac{9n-14}{4}$. Consequentemente, o problema inicial
      é $O(n)$.
  \end{enumerate}

	%%%% exercício 3 %%%%
	\item \begin{enumerate}

		\item Caso 1: $f(n) = 1 = O(n^{log_4 2})$, logo $T(n) =
		\Theta(\sqrt{n})$.

		\item Caso 2: $f(n) = \sqrt{n} = \Theta(n^{log_4 2})$, logo $T(n) =
		\Theta(\sqrt{n}\lg n)$.

		\item Caso 3: $f(n) = n = \Omega(n^{log_4 2})$ e $2\frac{n}{4} \leq c
		\cdot n$ para $c=1$, logo $T(n) = \Theta(n)$.

		\item Caso 3: $f(n) = n^2 = \Omega(n^{log_4 2})$ e $2\frac{n^2}{16}
		\leq c \cdot n^2$ para $c=1$, logo $T(n) = \Theta(n)$.

	\end{enumerate}

	%%%% exercício 4 %%%%
	\item Essa recorrência se encaixa no caso 2 do teorema mestre, $f(n) =
	\Theta(1) = \Theta(n^{\log_2 1 = 0})$, logo, $T(n) = \Theta(n^{\log_2 1}
	\lg n) = \Theta(\lg n)$.

	%%%% exercício 5 %%%%
	\item Não pode ser aplicado o teorema mestre aqui, pois, $n^{\log_2 4} =
	n^2$ não é polinomialmente diferente de $n^2 \lg n$, supondo que essa
	recorrência caia no caso 3 do teorema, $n^{2+\epsilon}$ é assintoticamente
	maior que $n^2 \lg n$ para qualquer $\epsilon > 0$.

\end{enumerate}

\end{document}

